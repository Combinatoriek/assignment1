\documentclass[12pt]{article}

\usepackage{amsmath}
\usepackage{enumerate}
\usepackage[margin=5em]{geometry}

\title{Assignment 1}

\author{
	Hendrik Werner s4549775
}

\begin{document}
\maketitle

\section*{9}
\begin{enumerate}[a]
	\item %a
	Every $7^{th}$ number is divisible by 7. There are $999 - 100 = 899$ numbers in the range $[100, 999]$. This means that there are $\lfloor 899 / 7 \rfloor = 128$ numbers divisible by 7 in this range.
	\setcounter{enumi}{3} \item %d
	$\dfrac{3}{4}$ of all natural numbers are not divisible by 4. There are $\lceil \dfrac{3 * 899}{4} \rceil = 675$ numbers in the range $[100, 899]$ which are not divisible by 4.
	\setcounter{enumi}{6} \item %g
	Every $3^{rd}$ number is divisible by 3. There are $\lfloor 899 / 3 \rfloor = 299$ numbers divisible by 3 in the range $[100, 899]$.

	Since we do not want to count the numbers divisible by 4 as well, we subtract the numbers which are divisible by $lcm(3, 4) = 12$. There are $\lfloor 899 / 12 \rfloor = 74$ numbers divisible by 12 in the range $[100, 899]$.

	The final answer are all numbers divisible by 3, without the numbers divisible by 4 as well: $299 - 74 = 225$.
\end{enumerate}

\section*{10}

\section*{11}
\begin{enumerate}[a]
	\item %a
	\item %b
	\item %c
\end{enumerate}

\section*{12}

\section*{13}
\begin{enumerate}[a]
	\item %a
	We can prove the existence of such a number with the pigeonhole principle. We make 19 buckets of numbers according to their equivalence class $\mod 19$.

	We take 20 numbers consisting of only 7s: $7, 77, 777, \dots, 77777777777777777777$ and put them into their corresponding bucket. At this point, according to the pigeonhole principle, there is at least one bucket with at least two numbers in it.

	We look for a bucket with 2 numbers and subtract the smaller number from the bigger one. We are left with a number of the form $7 \dots 0 \dots \equiv 0 (\mod 19)$, which has a maximum length of 20 digits.

	\item %b
\end{enumerate}

\end{document}
